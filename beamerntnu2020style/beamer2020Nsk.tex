\documentclass[aspectratio=169,mathserif]{beamer}
\usepackage[norsk]{babel}
\usepackage{xcolor}
\usepackage{transparent}
\usepackage[default,scale=1]{opensans}
\usepackage[T1]{fontenc}
\usepackage[utf8]{inputenc}

\mode<presentation>
{
	\usefonttheme{structurebold} 
	% Fjern det følgende for å vise navigasjonssymboler
	\setbeamertemplate{navigation symbols}{}
	% Konstruksjon av rammetittelen
	\setbeamertemplate{frametitle}{% 
		\kern1em\hskip-15pt
		\usebeamercolor[fg]{section}% 
		\usebeamerfont{section}% 
		\insertsection \hspace{0,1em} - {\normalsize \insertframetitle}
	}
	% Konstruksjon av fotlinjen
	\setbeamertemplate{footline}{% 
		\kern1em\hskip3em% 
		\includegraphics[width=0.3\textwidth]{ntnulogo_nsk.png}
		\hfill% 
		\usebeamercolor[fg]{page number in head/foot}% 
		\usebeamerfont{page number in head/foot}% 
		\insertframenumber%
		% Fjern merket med den følgende linjen for å vise det totale antallet sider i fotnoten
		%\,/\,\inserttotalframenumber
		\hskip12pt%
		\kern1.5em\vskip2em% 
	}
	% Definerer skrifter
	\setbeamerfont{title}{shape=\bfseries, size=\huge}
	\setbeamerfont{subtitle}{series=\mdseries,size=\Large}
	
	% Definerer fargene. Her kan du finne flere elementer hvis farge kan endres:
	%http://www.cpt.univ-mrs.fr/~masson/latex/Beamer-appearance-cheat-sheet.pdf
	\definecolor{NTNUBlue}{HTML}{00509e}
	\definecolor{LightGrey}{HTML}{D3D3D3}
	\setbeamercolor{background canvas}{bg=NTNUBlue}
	\setbeamercolor{title}{fg=white}
	\setbeamercolor{subtitle}{fg=white}
	\setbeamercolor{date}{fg=LightGrey}
	\setbeamercolor{author}{fg=LightGrey}
	\setbeamercolor{frametitle}{fg=NTNUBlue}
	\setbeamercolor{itemize item}{fg=NTNUBlue}
	\setbeamercolor{enumerate item}{fg=NTNUBlue}
	\setbeamercolor{block title}{fg=NTNUBlue}
	\setbeamercolor{itemize subitem}{fg=NTNUBlue}
	\setbeamercolor{enumerate subitem}{fg=NTNUBlue}
} 
	\usepackage{hyperref}
	\hypersetup{
	colorlinks=true,% make the links colored
	linkcolor=NTNUBlue,
	urlcolor=NTNUBlue
	}

% Gjennomføring av den endelige siden
\AtEndDocument{\begin{frame}[plain, noframenumbering]
		\begin{center}
			\vspace{4em}
			{\huge Takk for deres oppmerksomhet}\\
			\vspace{5em}
			\includegraphics[width=0.7\textwidth]{ntnulogo_nsk.png}
		\end{center}
	\end{frame}
}

% ----------> Skriv her innholdet på forsiden <----------
\title[Your Short Title]{Åpningsside/tittel her}
\subtitle{Undertittel. Evt. navn/dato/årstall}
\institute{\includegraphics[width=0.7\textwidth]{ntnulogo_nsk_neg.png}}
% Husk å fjerne de 3 linjene i neste avsnitt for å vise forfatteren
\author{Name of the author}
% Husk å fjerne de 3 linjene i neste avsnitt for å vise datoen
\date{\today}

\begin{document}
	
	%---------------------------------- Konstruerer forsiden ----------------------------------
	\begin{frame}[plain, noframenumbering]
		\vfill
		\centering	
		\begin{beamercolorbox}[sep=8pt,center,colsep=-4bp,rounded=true,shadow=true]{institute}
			\usebeamerfont{institute}\insertinstitute
		\end{beamercolorbox}	
		\vskip2.5em\par
		{\usebeamercolor[fg]{titlegraphic}\inserttitlegraphic\par}	
		\begin{beamercolorbox}[sep=8pt,center,colsep=-4bp,rounded=true,shadow=true]{title}
			\usebeamerfont{title}\MakeUppercase{\inserttitle}\par%
			\ifx\insertsubtitle\@empty%
			\else%
			\vskip0.5em%
			{\usebeamerfont{subtitle}\usebeamercolor[fg]{subtitle}\insertsubtitle\par}%
			\fi%     
		\end{beamercolorbox}%
		
		% ---------->	Fjern det følgende for å vise forfatteren <----------
		%	\begin{beamercolorbox}[sep=8pt,center,colsep=-4bp,rounded=true,shadow=true]{author}
		%		\usebeamerfont{author}\insertauthor
		%	\end{beamercolorbox}
		
		% ---------->	Fjern det følgende for å vise datoen <----------
		%	\begin{beamercolorbox}[sep=8pt,center,colsep=-4bp,rounded=true,shadow=true]{date}
		%		\usebeamerfont{date}\insertdate
		%	\end{beamercolorbox}\vskip0.5em
		
		% ---------->	Hvis du legger til forfatter eller dato, reduserer du følgende mellomrom <----------
		\vskip4em
		
	\end{frame}
	
	% Innstiller bakgrunnsfargen til ingen
	\setbeamercolor{background canvas}{bg=}
	
	% ---------->	Fjern kommentarene for en automatisk generert innholdsfortegnelse. <----------
	%\begin{frame}{Outline}
	%  \tableofcontents
	%\end{frame}
	
	%----------------------------------Eksempel av seksjonen: Introduksjon----------------------------------
	%Navnet på seksjonen, for referanser eller innholdsfortegnelse
	\section{Introduksjon}
	
	\begin{frame}{Et typisk lysbilde}
		
		
		\begin{itemize}
			\item Skriv introduksjonen her.
			\item Bruk \texttt{itemize} eller \texttt{enumerate} for å organisere hovedpoengene.
		\end{itemize}
		
		\vskip 1cm
		
		\begin{block}{Exsempel}
			I stedet for fritt å skrive tekst i rammen, bruk blokker som dette.
		\end{block}
		
	\end{frame}
	
	
\end{document}
