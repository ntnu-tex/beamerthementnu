\documentclass[aspectratio=169]{beamer}
\usepackage[english]{babel}
\usepackage{booktabs,listings}
\usepackage[T1]{fontenc}
\usepackage[utf8]{inputenc}
\lstset{basicstyle=\ttfamily}

\usetheme[slogan=english, style=stripe]{NTNU}
%
% Edit your meta data here
%
	\title{\LaTeX{}-Beamer Style for NTNU}
	\subtitle{Documentation.}
	\author{Ronny Bergmann}
	\date{\today}

\begin{document}
	\maketitle
	\tableofcontents

	\section{Introduction}
	\begin{frame}[fragile]{The available options}
		\begin{tabular}{lcl}
			\toprule
			\textbf{Option} & \textbf{Default} &\textbf{Description} \\
			\midrule
			\lstinline!displayframetotal! && the same as \lstinline!!frametotal=true! \\
			\lstinline!frametotal=true|false! & \lstinline!false! & toggle display the total number of slides\\
			\lstinline!hideframetotal! && the same as \lstinline!!frametotal=false! \\
			\lstinline!slogan=english|norsk! & \lstinline!norsk! & whether to display english or norsk slogan\\
			\lstinline!style= ...! & \lstinline!plain! & styles, see slide \ref{slide:outer}\\
			\bottomrule
		\end{tabular}
	\end{frame}
	\begin{frame}{Outer themes}
		\label{slide:outer}
		The corporate design team provides four variants (outer themes in latex beamer speaking)

		\begin{description}
			\item[NTNUcircles] using circles in the background (not yet available)
			\item[NTNUplain] a very plain style (Default)
			\item[NTNUstripe] with a stripe on the left (current)
			\item[NTNUstripebottom] with a stripe at the bottom
		\end{description}

		While you can set this using e.g. \lstinline!\\useoutertheme\{NTNUcircles\}! and \lstinline!\\useinnertheme\{NTNUcircles\}!
		in your preamble, the easier way is the option
        \lstinline!style=circles|plain|stripe|stripebottom!
	\end{frame}
	\begin{frame}{special commands}
		\begin{description}
			\item[titlelogo]
			Set the logo on the title page. By default it is set to the negative english or norsk logo (depending on the chosen slogan), see the slogan option
		\end{description}
	\end{frame}
	\begin{frame}{Special colors}
%		These colors are taken from \url{https://innsida.ntnu.no/wiki/-/wiki/Norsk/Farger+i+grafisk+profil}
		\begin{description}
			\item[{\color{NTNUBlue} NTNUblue}]
			\item[{\color{NTNULightblue} NTNULightblue}]
			\item[{\color{NTNUOrange} NTNUOrange}]
			\item[{\color{NTNUPink} NTNUPink}]
			\item[{\color{NTNUYellow} NTNUYellow}]
			\item[{\color{NTNUViolet} NTNUViolet}]
			\item[{\color{NTNUCyan} NTNUCyan}]
			\item[{\color{NTNUOcher} NTNUOcher}]
		\end{description}
	\end{frame}
	\begin{frame}{Required packages}
		The following packages have to be installed for the theme to work
		\begin{description}
			\item[calc] for a few computational tricks
			\item[ifthen] to define some commands
			\item[pdftexcmds] to define some commands
			\item[opensans] for the font
			\item[tikz] for some graphic tricks
		\end{description}
		\ \\[\baselineskip]
		This documentation further uses
		\begin{description}
			\item[booktabs] for nice tables
			\item[listings] for code highlighting
		\end{description}
	\end{frame}
\end{document}
